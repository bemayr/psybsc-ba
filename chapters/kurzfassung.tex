\chapter{Kurzfassung}
\todo{that's new}
\noindent
Die Entwicklung von Software ist eine äußerst menschenzentrierte Angelegenheit, die Aufmerksamkeit die ihr die Psychologie dabei schenkt, ist im Gegensatz verschwindend gering, was durchaus am sehr großen Unterschied dieser zwei Wissenschaftsbereiche liegt.
Viel zu oft wird Programmieren auf das reine Erstellen von Quelltexten reduziert, oder umgangssprachlich gar als Tippen von Nullen und Einsen bezeichnet.
In Wirklichkeit sind die Prozesse, die zum Erstellen von Software notwendig sind allerdings wesentlich komplexer und interdisziplinärer.
Das Ziel dieser Arbeit ist die Betrachtung des Softwareentwicklungsprozesses aus der psychologischen Perspektive, wobei der Fokus auf der Erstellung und Weitergabe von mentalen Modellen und gemeinsamen Verständnis liegt.
Die Forschungsfrage lautet daher: \emph{Durch die Kombination von Statecharts und Hole-Driven-Development entsteht eine neue Art und Weise Software zu entwickeln, welche Erkenntnisse der Kognitions- und Sozialpsychologie unterstützen diese Kombination, bzw. wie kann diese so adaptiert werden, dass sie den Prozessen unseres Gehirns ähnelt?}

\vspace{2cm}
\noindent
Keywords: Mentale Modelle, Psychologie der Computerprogrammierung, Statecharts