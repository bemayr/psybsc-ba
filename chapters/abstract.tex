\chapter{Abstract}
\todo{that's new}
\noindent
Despite of being a human activity, developing software usually does not receive the amount of psychological research needed.
A major difficulty is the huge gap between those two disciplines being the reason for this missing interdisciplinary thinking.
Developing software is way too often reduced to the act of entering commands in a way the computer does what the programmer wants it to do.
In reality a lot of steps happen until software developers can express their ideas in such a clarity that they are able to create an executable and correct program.
Looking at the software development process holistically and applying ideas of cognitive psychology, especially these of mental models, the author of this thesis tries to align the thought process behind creating software closer with the actual technical process of doing it.
This leads to the research question: \emph{Which findings of cognitive and social psychology support the usage of statecharts combined with hole-driven development for enabling better communication in software development processes, or how can these concepts be adapted to closer align to the way the human brain works?}

\vspace{2cm}
\noindent
Keywords: mental models, psychology of programming, statecharts