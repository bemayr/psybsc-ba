\chapter{Abstract}
- "A programmer does not primarily write code; rather, he primarily writes to another programmer about his program solution." \autocite{noauthor_what_1967}
Despite of being a human activity, developing software usually does not receive the amount of psychological research needed. Probably the huge gap between those two disciplines is the reason for this missing interdisciplinary thinking. Developing software is way too often reduced to the act of entering commands in a way the computer does what the programmer wants it to do. In reality a lot of steps happen until software developers can express their ideas in such a clarity that they are able to create an executable and correct program. A lot of steps take place until this final one. Looking at the software development process holistically and applying ideas of cognitive psychology, especially those of mental models, the author of this thesis tries to align the thought process behind creating software closer to the actual process of doing it.

This results in the current working title: \emph{A psychological exploration on how the interactivity of Statecharts combined with hole driven development could lead to a common level of communication for all participants involved in software development processes.}