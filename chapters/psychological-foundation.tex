\chapter{Psychological Foundation (14)}

\section{Psychology of Programming (4)}
- "The origins of PoP date back to late 1970s and early 1980s, when researchers realized that programming tools and technologies should not be evaluated based on their computational power only, but also on their usability from the human point of view, that is, based on their cognitive effects." \autocite[4]{sajaniemi_psychology_2008}
- Motivation of Computer Scientists vs. Psychologists \autocite[4]{sajaniemi_psychology_2008}
- 2READ: \autocite{detienne_expert_1990}
- Assistance to the design activity \autocite[246--247]{visser_expert_1990}
- 2READ: \autocite{watt_syntonicity_1998}
- \autocite{weinberg_psychology_1971}
\subsection{Programming as Theory Building}
- \autocite{naur_programming_1985}
- \url{https://catenary.wordpress.com/2011/04/19/naurs-programming-as-theory-building/}
- \url{https://twitter.com/ruthmalan/status/1245058319435300866}
- \url{https://twitter.com/ruthmalan/status/1144563675128303616}
\subsection{Metaphors}
- \autocite[239--240]{naur_programming_1985}
- METAPHORS WE COMPUTE BY - Alvaro Videla
  - \url{https://alvaro-videla.com/2017/01/metaphors-we-code-by.html}
  - \url{https://hooktube.com/watch?v=hKOzJWNRBWA&feature=youtu.be}
  - "Sometimes our tools do what we tell them to do. Other times, we adapt ourselves to our tools’ requirements" - Nicholas Carr

\section Visual Notations
- Introduction + "picture superiority effect" \autocite[756]{moody_physics_2009}
- "Visual notations are uniquely human-oriented representations: Their sole purpose is to facilitate human communication and problem solving" \autocite[757]{moody_physics_2009}
- Why Visual Representation Is Important \autocite[758--759]{moody_physics_2009}
- "Spatial Fashion in Smalltalk" [Bret Victor - The Future of Programming](https://hooktube.com/watch?v=8pTEmbeENF4) (19:34)
- [Image: The Danger of Visual Representation](https://twitter.com/parport0/status/1253716518363369475?s=21)
- [Visual Representation - The Encyclopedia of Human-Computer Interaction, 2nd Ed.](https://www.interaction-design.org/literature/book/the-encyclopedia-of-human-computer-interaction-2nd-ed/visual-representation)
\subsection{Dual Coding Theory}
- "In a position in-between we can cite Agnyal [1], who claims that using two
synchronised notations (graphical-textual) for the domain model improves its
maintainability." \autocite[3]{melia_comparison_2016}
- In regards to Statecharts and modifications \autocite[38--39]{leveson_experiences_1991}
- \url{https://en.wikipedia.org/wiki/Dual-coding_theory}

\section{Solving Complex Problems (3)}
- Complexity \autocite[5--11]{ousterhout_philosophy_2018}
- "Ill-defined problems" \autocite[236]{visser_expert_1990}
- Programming Languages: Different Levels of analysis and flow of development (\ref{sec:hole-driven}) \autocite[242--244]{visser_expert_1990}
\subsection{In Groups}

\section Mental Models (5)
- Cognitive Models (\ref{sec:applications}) \autocite[276]{kitchenham_research_1990}
- Cognitive Issues in Software Development regarding Design Notations (Thinking in States, Mathematical vs. Structured Methods) \autocite[281--282]{kitchenham_research_1990}
- 2READ: \autocite{gentner_mental_2014}
- 2READ: \autocite{dutke_mentale_1994}
- 2READ: \autocite{herczeg_software-ergonomie_2018}
- 2READ: \autocite{wandmacher_software-ergonomie_1993}
\subsection{Sharing Mental Models}
- Transferring Mental Models "The specification must be formal enough for us to use as a basis for a safety analysis. It must also be readable enough for noncomputer experts to read and review and be usable for both building and certifying TCAS I1 systems." \autocite[32]{leveson_experiences_1991}
- Communicating Mental Models \autocite[275]{kitchenham_research_1990}
- Sharing Designs \ref{sec:literate-programming} \autocite[282]{kitchenham_research_1990}
- Transferring Knowledge \autocite[234--235]{naur_programming_1985}
\subsection{Cross-Domain}
- Mental Models in Statecharts \autocite[2]{harel_statecharts_2007}
\subsection{Human-Computer vs. Human-Human Interaction (Literate Programming)}
- \url{https://en.wikipedia.org/wiki/Literate_programming}

\section{Social Aspects of Team Communication (2)}