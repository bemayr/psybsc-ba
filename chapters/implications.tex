\chapter{Implications (10)}

The software development process involves a lot of participants. The customer or user imagines the perfect product, the requirements engineer thinks in user stories, testers use acceptance criteria and developers think in source code. All of them create their own mental model of the same thing. When building something like a virtual reality training in addition to the previously mentioned participants there are content designers who think in flowcharts and narrators who have narration structures in their minds when thinking and talking about the training.
They have all built up their way of approaching the problem and also use the tools appropriate to their task, but talking about the same thing in "different languages" always requires translations. So why not build a system that does these translations and takes this cognitive load off of people?



- Thinking Tools \autocite[273--274]{kitchenham_research_1990}

\section{Building Prototypes}
- Simulation aspects of Statecharts \autocite[7]{harel_statecharts_2007}

\section{Finding Errors}

\section{Analyzing Requirements}
- \autocite{leveson_experiences_1991}

\section{Communicating Change Requests}
- \ref{sec:hole-driven-development}
- Mental Gap using Issue Tracking Systems

\section{Gradual Problem Solving}
- \ref{sec:hole-driven-development}
- \ref{sec:fractal-nature}