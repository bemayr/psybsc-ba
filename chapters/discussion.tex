\chapter{Discussion}
\label{chap:discussion}
\epigraph{``One way to define `programming' is as the process of transforming a mental plan of desired actions for a computer into a representation that can be understood by the computer.''}{\textit{Jean-Michel Hoc}}
\todo{TODO}
- Structure not Flow \autocite{melia_comparison_2016}
- "I have concentrated on requirements models only. The applicability and usability of the proposed statechart variant for other purposes, in particular for architecture and detailed design remains to be investigated." \autocite[5]{glinz_statecharts_2002}
- A critical view on Statecharts \autocite{breen_statecharts_2004}
- State vs. Context \autocite[36]{leveson_experiences_1991}
- Why Statecharts aren't used more widely \autocite[9]{harel_statecharts_2007}
- Flowcharts vs. Statecharts (they are a model) \autocite[10]{harel_statecharts_2007}

- "Consequently, this approach can help to avoid the waste problem that results in redevelopment effort from incorrectly specified products." \autocite[10]{sheldon_software_2000}
- [Matthew Frederick - three levels of knowing](https://twitter.com/Kpaxs/status/1115633791966511109)

risk of different mental models created that are not understood by colleagues

\section{Further Implications}
\subsection{Building Prototypes}
\subsection{Thinking in States/Preventing Errors}
\subsection{Analyzing Requirements}
- \autocite{leveson_experiences_1991}

\section{Literate Programming}
%- Human-Computer vs. Human-Human Interaction
%\label{sub:literate-programming}
%- Communicating Mental Models \autocite[275]{kitchenham_research_1990}
%- \url{https://en.wikipedia.org/wiki/Literate_programming}
%- \autocite[235]{naur_programming_1985}
%- METAPHORS WE COMPUTE BY - Alvaro Videla
%  - \url{https://alvaro-videla.com/2017/01/metaphors-we-code-by.html}
%  - \url{https://hooktube.com/watch?v=hKOzJWNRBWA&feature=youtu.be}