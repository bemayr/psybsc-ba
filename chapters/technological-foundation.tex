\chapter{Technological Foundation (11)}

\section{Characteristics of Software Development (Processes) (3)}
- \autocite{mayr_projekt_2005}
- Strategic vs. Tactical Programming \autocite[13--18]{ousterhout_philosophy_2018}
- Bottom-Up-Approach \autocite[17]{horrocks_constructing_1999}
- Layered Behavioural Model (human centered) \autocite[254]{curtis_psychology_1990}
- Boundary Spanners \autocite[264]{curtis_psychology_1990}
- CITE: what makes the software development process special \autocite[274]{kitchenham_research_1990}
- "any documentation being an auxiliary, secondary product" \autocite{naur_programming_1985}
- Programming as a Social Activity \autocite{weinberg_psychology_1971}
\subsection{Traditional}
- Developing Software does not happen Top-Down, but the conceptual models are closely related to the Waterfall Model \ref{sec:mental-models} \autocite[275]{kitchenham_research_1990}
\subsection{Agile}
- On why agile can't work (\ref{sec:programming-as-theory-building}) \autocite[232--233]{naur_programming_1985}
\subsection{Requirements Engineering and Changing Requirements}
- Requirements Specification \autocite[32]{leveson_experiences_1991}
- Information Loss \autocite[265]{curtis_psychology_1990}

\section{Statecharts (3)}
- Introduction \autocite[49]{horrocks_constructing_1999}
- 2READ: \autocite{harel_statecharts:_1987}
- State vs. Context \autocite[36]{leveson_experiences_1991}
- Ideas behind Statecharts \autocite[1]{harel_statecharts_2007}
- Reactive Systems \autocite[2]{harel_statecharts_2007}
- Goal \autocite[3]{harel_statecharts_2007}
- "I became convinced from the start that the notion of a state and a transition to a new state was fundamental to their thinking about the system." \autocite[4]{harel_statecharts_2007}
- where the name comes from \autocite[4]{harel_statecharts_2007}
- Mathematical Ideas behind them \autocite[6]{harel_statecharts_2007}
- Simulation aspects of Statecharts \autocite[7]{harel_statecharts_2007}
\subsection{Aligning Thoughts with Fractal Nature}
- Bottom-Up-Approach \autocite[19--25]{horrocks_constructing_1999}
- Internal Model \autocite[33]{leveson_experiences_1991}

\section{Hole Driven Development (2)}
- Holes \autocite[20--21]{brady_type-driven_2017}
- \url{http://bitemyapp.com/posts/2017-09-23-please-stop-using-typed-holes.html}
- \url{http://vaibhavsagar.com/blog/2017/05/22/discovering-continuations/}
- \url{https://matthew.brecknell.net/posts/hole-driven-haskell/}
- \url{https://wiki.haskell.org/GHC/Typed_holes}
- \url{https://github.com/maddogdavis/holey}
- \url{https://ghc.haskell.org/trac/ghc/wiki/Holes}
- \url{https://www.shimweasel.com/2015/02/17/typed-holes-for-beginners}
- \url{https://www.reddit.com/r/haskell/comments/10u7xr/ghc_head_now_features_agdalike_holes/c6h5mev/}
- \url{http://docs.idris-lang.org/en/latest/tutorial/typesfuns.html#holes}
\subsection{Incomplete Thoughts}
- "Adelson et al. (1985) observer their experts making 'notes to themselves'" \autocite[241]{visser_expert_1990}

\section{Visual Programming (3)}
- "information exposure" \autocite[34]{leveson_experiences_1991}
- Readability and Reviewability \autocite[37-38]{leveson_experiences_1991}
- Some Notes on Visual Programs \autocite[5--6]{green_pictures_1982}
- Flowcharts $\rightarrow$ Statecharts (\ref{sec:statecharts}) \autocite[8]{green_pictures_1982}
- "backwards tracing is easier in graphical notation than in programming languages." \autocite[20]{green_pictures_1982}
\subsection{History}
\subsection{Problems}
- \autocite{melia_comparison_2016} Only examined research on one specific notation.
    - "Petre [36] points at the fact that, since using textual modelling languages is similar to programming, the learning curve is lower than for graphical representations" \autocite[3]{melia_comparison_2016}
    - "Subjects in the experiment performed significantly better both for analysability coverage and modifiability efficiency with a textual notation" \autocite[26]{melia_comparison_2016}
    - "The most relevant threat to the validity of our results is that the quasi-experiment was performed with students and small models" \autocite[26]{melia_comparison_2016}
    - Junior Developers, Trained using Textual Notation
- on putting things in boxes (subroutines) \autocite[23]{green_pictures_1982}
- 2READ: \autocite{bresciani_pitfalls_2015}
- \autocite{cook_unified_2017}