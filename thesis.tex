\documentclass[a4paper, bibliography=totoc, oneside, 12pt]{scrbook}
\usepackage[american]{babel}
\usepackage{csquotes}
\usepackage[style=apa,sortcites=true,sorting=nyt,backend=biber]{biblatex}
\DeclareLanguageMapping{american}{american-apa}
\addbibresource{bibliography.bib}

\usepackage{enumitem} % numbered lists
\usepackage{ragged2e} % justification
\usepackage{textcomp}
\usepackage{setspace}
\usepackage{microtype} % fixes line breaks
\usepackage[a4paper, margin=2.5cm, top=3.5cm, bottom=3.5cm]{geometry}
\usepackage[automark, headsepline, singlespacing=true]{scrlayer-scrpage}

% \setlength{\headheight}{15.35403pt} % Warning: headheight too small
\linespread{1.25}
%\setlength\parindent{0pt}
%\setlength{\parskip}{12pt}

\listfiles

\title{Aligning Mental Models and Models Used in Software Development}
\author{Bernhard Mayr}

\clearpairofpagestyles
\ohead*{\pagemark}


\begin{document}

\makeatletter
\begin{titlepage}
   \begin{center}
        Institute for Psychology\\
        Faculty for Psychology und Sport Science\\
        Leopold--Franzens--University Innsbruck
 
        \vspace{4cm}
        \textit{\textbf{\Large {\@title}}}
 
        \vspace{2cm}
        Bachelor's Thesis\\
        for obtaining the Academic Degree\\
        Bachelor of Science (B.\,Sc.)\\
        in the academic field Psychology
 
        \vspace{3cm}
        submitted by\\
        \textit{\@author, B.\,Sc., 11730144}
 
        \vfill
 
        Supervisor: \textit{Univ.-Prof. Dr. Pierre Sachse}\\
        Innsbruck, \textit{\today}
   \end{center}
\end{titlepage}

\frontmatter
\setcounter{page}{2}
\tableofcontents
\mainmatter

\chapter{Abstract}
Despite of being a human activity, developing software usually does not receive the amount of psychological research needed. Probably the huge gap between those two disciplines is the reason for this missing interdisciplinary thinking. Developing software is way too often reduced to the act of entering commands in a way the computer does what the programmer wants it to do. In reality a lot of steps happen until software developers can express their ideas in such a clarity that they are able to create an executable and correct program. A lot of steps take place until this final one. Looking at the software development process holistically and applying ideas of cognitive psychology, especially those of mental models, the author of this thesis tries to align the thought process behind creating software closer to the actual process of doing it.

This results in the current working title: \emph{A psychological exploration on how the interactivity of Statecharts combined with hole driven development could lead to a common level of communication for all participants involved in software development processes.}


\chapter{Hypothesis and Research Question}
The process of developing software can be seen as a communication process. Beginning with requirements analysis, these have to be understood, implemented, tested and shipped to the customer. This process ideally happens iteratively and incrementally, which means that the steps are repeated and refined over time.
Currently the implementation of this process entails a lot of problems as described in the next section.
Utilizing the features of Statecharts and hole driven development to align this process closer to how people think might result in a huge productivity boost and take a lot of burden off the people involved in creating software.

\verb+\autocite{blackwell_logical_2010}+ \textrightarrow{} \textbf{\autocite{blackwell_logical_2010}}

\verb+\textcite{blackwell_logical_2010}+ \textrightarrow{} \textbf{\textcite{blackwell_logical_2010}}

Statecharts \autocite{harel_statecharts:_1987} are a "visual formalism for complex systems". In more general words this means that Statecharts describe a mathematically proven visual language for creating computer programs. The visual notation is executable and can directly be aligned to a textual notation, which results in the fact, that a program can be visualized automatically. This visualization combines flowcharts and state diagrams. On top of that this visualization provides simulation capabilities and therefore enables people to visually see and experiment with the \emph{logic} of the program, without actually using the program. In fact, the program might not even exist at this point, but people can collaboratively design \emph{what the program should do.} The main hypothesis is that this duality of visualization and executability enables the creation of a common mental model, resulting in a common level of communication.

To further improve this communication process, the fractal nature of Statecharts (logic can be refined from abstract to concrete) is the perfect fit for hole driven development. This relatively new idea of writing computer programs can be compared with a "fill in the blank" exercise. Instead of having to write a complete program, developers can leave holes for the computer and then ask it to provide help. After that, mathematical reasoning combined with knowledge about the already written parts can be applied to provide proposals for these blanks. Applying this idea at a way higher level, the second hypothesis is that using holes in Statecharts (enabled by their fractal nature) enables a new way of communicating and refining requirements.

This leads to the research question: \emph{Which findings of cognitive psychology support the usage of Statecharts combined with hole driven development for enabling better communication in software development processes, or how can those concepts be adapted to closer align to the way the human brain works?}

% Code is the *real* source of truth


\section{State of the Art}
\label{sec:state-of-the-art}
The software development process involves a lot of participants. The customer or user imagines the perfect product, the requirements engineer thinks in user stories, testers use acceptance criteria and developers think in source code. All of them create their own mental model of the same thing. When building something like a virtual reality training in addition to the previously mentioned participants there are content designers who think in flowcharts and narrators who have narration structures in their minds when thinking and talking about the training.
They have all built up their way of approaching the problem and also use the tools appropriate to their task, but talking about the same thing in "different languages" always requires translations. So why not build a system that does these translations and takes this cognitive load off of people?


\section{Structure}
\begin{enumerate}[label*=\arabic*.]
    \item Problem Definition
    
    \item Technological Foundation
    \begin{enumerate}[label*=\arabic*.]
        \item Characteristics of Software Development Processes
        \item Statecharts % fractal nature, aligning
        \item Hole Driven Development % incomplete thoughts
        \item Visual Programming
    \end{enumerate}
    
    \item Psychological Foundation
    \begin{enumerate}[label*=\arabic*.]
        \item Mental Models
        \item Solving Complex Problems
    \end{enumerate}
    
    \item Applications
    \begin{enumerate}[label*=\arabic*.]
        \item Building Prototypes
        \item Finding Errors
        \item Analyzing Requirements
        \item Communicating Change Requests
        \item Gradual Problem Solving
    \end{enumerate}
    
    \item Outlook
\end{enumerate}


\section{Approach and Methodology}
The way I approach the writing of this thesis is a combination of literature review and implementing and exploring the ideas in my job at Innerspace GmbH, Wattens.
My current position there is Senior Researcher concerning the topic of Statecharts.
Up to now I did a lot of research concerning the process of how software (virtual reality trainings in this case) is developed and, based on that, carried out the technical implementation of Statecharts.

Over the next few months my task at work is to implement the formalism of Statecharts as the basis of our software development process. The knowledge gained within this first step will be directly included in this thesis and used for further research. Further the research done in the context of writing this thesis will be used proactively to improve the way I introduce Statecharts into the software development process.

As a conclusion the methodology can be defined as a literature review with its results being tested directly in practice, and these results again used as the basis for further research.


\addchap{Preliminary References}
\nocite{*}
\printbibliography[heading=none]

\end{document}
